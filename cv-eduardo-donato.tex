\documentclass[]{article}
\usepackage{lmodern}
\usepackage{amssymb,amsmath}
\usepackage{ifxetex,ifluatex}
\usepackage{fixltx2e} % provides \textsubscript
\ifnum 0\ifxetex 1\fi\ifluatex 1\fi=0 % if pdftex
  \usepackage[T1]{fontenc}
  \usepackage[utf8]{inputenc}
\else % if luatex or xelatex
  \ifxetex
    \usepackage{mathspec}
  \else
    \usepackage{fontspec}
  \fi
  \defaultfontfeatures{Ligatures=TeX,Scale=MatchLowercase}
\fi
% use upquote if available, for straight quotes in verbatim environments
\IfFileExists{upquote.sty}{\usepackage{upquote}}{}
% use microtype if available
\IfFileExists{microtype.sty}{%
\usepackage{microtype}
\UseMicrotypeSet[protrusion]{basicmath} % disable protrusion for tt fonts
}{}
\usepackage[unicode=true]{hyperref}
\hypersetup{
            pdfborder={0 0 0},
            breaklinks=true}
\urlstyle{same}  % don't use monospace font for urls
\usepackage{longtable,booktabs}
\IfFileExists{parskip.sty}{%
\usepackage{parskip}
}{% else
\setlength{\parindent}{0pt}
\setlength{\parskip}{6pt plus 2pt minus 1pt}
}
\setlength{\emergencystretch}{3em}  % prevent overfull lines
\providecommand{\tightlist}{%
  \setlength{\itemsep}{0pt}\setlength{\parskip}{0pt}}
\setcounter{secnumdepth}{0}
% Redefines (sub)paragraphs to behave more like sections
\ifx\paragraph\undefined\else
\let\oldparagraph\paragraph
\renewcommand{\paragraph}[1]{\oldparagraph{#1}\mbox{}}
\fi
\ifx\subparagraph\undefined\else
\let\oldsubparagraph\subparagraph
\renewcommand{\subparagraph}[1]{\oldsubparagraph{#1}\mbox{}}
\fi

\date{}

\begin{document}

\begin{longtable}[]{@{}lll@{}}
\toprule
\textbf{Eduardo Batista Donato} &
\href{mailto:eduardo.donato@gmail.com}{\nolinkurl{eduardo.donato@gmail.com}}
& Vitória - ES\tabularnewline
\midrule
\endhead
\textbf{(27) 99969-3809} & \url{https://br.linkedin.com/in/ebdonato} &
Casado, sem filhos\tabularnewline
& \url{https://github.com/ebdonato} & 28/08/1982\tabularnewline
\bottomrule
\end{longtable}

\section{OBJETIVO}\label{objetivo}

GESTÃO DE PROJETOS \textbar{} ENGENHARIA DE PLANEJAMENTO

\section{FORMAÇÃO}\label{formauxe7uxe3o}

Pós-graduação em ENGENHARIA DE PLANEJAMENTO -- Universidade Federal do
Espírito Santo / PROMINP (2007).

Graduação em ENGENHARIA ELÉTRICA -- Universidade Federal de Viçosa (MG,
2006).

Idioma Inglês avançado.

\section{RESUMO DE QUALIFICAÇÕES}\label{resumo-de-qualificauxe7uxf5es}

Experiência em gestão de projetos internos de uma planta industrial de
grande porte, incluindo o controle de cronograma, do escopo e da
qualidade, promovendo a integração entre os diversos setores da fábrica
e melhorando processos de fluxo de informações e de responsabilidades.

Experiência em gerenciamento e fiscalização de obras com ênfase na
disciplina planejamento, acompanhando e verificando o planejamento e os
avanços físicos e financeiros.

Experiência em desenvolvimento de equipamentos para medição e controle
industriais, incluindo programação alto nível (C++) e \emph{design} do
hardware.

Vivência avançada com aplicativos de produtividade (Microsoft Office,
incluindo Access, Power BI, VBA, Project, Sharepoint), e ferramentas de
desenvolvimento (wxWidgets, Microsoft Visual Studio, MySQL, PostgreSQL,
C++, MIT App Inventor, Javascript {[}VueJS, NodeJS, ExpressJS{]}, Html,
CSS).

Habilidades em extração e tratamento de dados/relatórios do SAP,
incluindo desenvolvimento~de macros (scripts) para processos
automatizados.

Desenvolvimento de soluções avançadas com Microsoft Power Platform
(Power Apps, Power Automate (Flow) e Power BI).

Experiência em liderança através da coordenação de uma equipe com dois
supervisores e quase 100 colaboradores.

\section{EXPERIÊNCIAS
PROFISSIONAIS}\label{experiuxeancias-profissionais}

\subsection{Analista de Desenvolvimento na Eletromarquez -- 04/2020
(atual)}\label{analista-de-desenvolvimento-na-eletromarquez-042020-atual}

Responsável pela automatização de processos rotineiros e/ou repetitivos
através do desenvolvimento de aplicativos e scripts (robôs).
Desenvolvimento de aplicativos em Excel (macro/VBA), relacionando
planilhas internas e relatórios do SAP, e de aplicativos para o usuário
final com Microsoft Power Platform (Power Apps e Sharepoint).
Responsável pela modelagem, desenvolvimento e implantação de um Sistema
Integrado de Gestão Empresarial (ERP) através de tecnologias web
modernas (Html, CSS, Javascript {[}VueJS, NodeJS e ExpressJS{]} e Banco
de Dados {[}PostgreSQL{]}). Responsável pelo desenvolvimento e
manutenção dos relatórios gerenciais dos contratos com Microsoft Power
BI.

\subsection{Analista de Planejamento na Time-Now Engenharia -- 08/2019 a
03/2020}\label{analista-de-planejamento-na-time-now-engenharia-082019-a-032020}

Gestão da cadeia de suprimentos da gerência de Projetos e Construção de
Alta Tensão da EDP. Modelagem, automatização, aperfeiçoamento e
implementação de processos na gestão da cadeia de suprimentos.
Responsável pelo desenvolvimento e manutenção dos relatórios gerenciais
do contrato com uso avançado da ferramenta Power BI, incluindo a
extração e o tratamento de dados do SAP. Desenvolvimento de macros para
interação com SAP. Avaliação e implementação da estratégia e das
ferramentas de gerenciamento de projetos.

\subsection{Coordenador de Operações na Eletromarquez -- 03/2017 a
07/2019}\label{coordenador-de-operauxe7uxf5es-na-eletromarquez-032017-a-072019}

Gestão de equipe de leituristas (medição do consumo e faturamento de
energia) a serviço da EDP, realizando a programação da distribuição da
equipe nas rotas diárias, a análise crítica da produtividade, da
efetividade e da qualidade das leituras e das faturas entregues aos
clientes da contratante e também promover e garantir que as atividades
sejam realizadas conforme planejado. Gestão de equipe de aproximadamente
100 colaboradores.

Responsável por um dos contratos com a EDP, liderando dois centros de
leitura (Cariacica/Viana e Vila Velha), com elaboração de relatórios de
\emph{status} e indicadores de performance contratuais, de qualidade e
de produção (KPI's), e gestão da equipe interna de apoio aos
colaboradores externos e dos procedimentos operacionais do contrato.

Desenvolvimento e aprimoramento de ferramentas computacionais para
automatização de processos (relatórios e índices) e otimização da
atividade fim, através de programação macro Excel (VBA) e aplicativos
Android.

\subsection{Engenheiro de Projetos na Flexibras Tubos Flexíveis
(TechnipFMC) -- 05/2013 a
11/2016}\label{engenheiro-de-projetos-na-flexibras-tubos-flexuxedveis-technipfmc-052013-a-112016}

Gestão de contratos, firmados ou em propostas, de fabricação de tubos
flexíveis recebendo as premissas, efetuando a análise crítica e enviando
para o planejamento da produção. Acompanhamento do desenvolvimento de
projeto, desde aquisição de matérias primas até a entrega ao cliente do
produto final e documentos relacionados, participando de reuniões de
planejamento, verificando e resolvendo os problemas eventuais, propondo
e aplicando ações preventivas e corretivas. Promover reuniões rotineiras
e extraordinárias com vários departamentos de uma ou mais unidades do
grupo, pontuando todas as possíveis dificuldades, mudanças, necessidades
e desafios do projeto. Interface com outras áreas e unidades do grupo,
no Brasil ou no exterior, visando solucionar problemas inerentes aos
projetos e estimar prazos e possibilidades de carga para a fábrica.
Gestão de conflitos de prioridade entre projetos de longo prazo,
apresentando soluções, dando sugestões de acordo com análise crítica do
planejamento, com o objetivo de definir os novos parâmetros.

\subsection{Engenheiro de Planejamento na Ductor Implantação de Projetos
(TÜV Rheinland) -- 11/2012 a
05/2013}\label{engenheiro-de-planejamento-na-ductor-implantauxe7uxe3o-de-projetos-tuxfcv-rheinland-112012-a-052013}

Gerenciamento e fiscalização das obras de manutenção e de melhoria da
Estrada de Ferro Vitória-Minas (EFVM), implantadas pela Vale S.A.
(GAIPG), gestão do OPEX, realizando o acompanhamento dos serviços e dos
avanços físico-financeiros, além da garantia do cumprimento das
especificações, das premissas, das exigências de segurança e saúde e da
qualidade do produto. Responsável por promover as reuniões de
planejamento e as reuniões de análises de riscos das obras (análise
SWOT) com as contratadas. Também a análise e a validação dos
cronogramas, estruturas analíticas de projeto (EAP), curvas de progresso
físico-financeiro, histogramas de mão-de-obra e equipamentos
apresentados.

\subsection{Engenheiro de Planejamento na MCA Auditoria e Gerenciamento
-- 06/2012 a
11/2012}\label{engenheiro-de-planejamento-na-mca-auditoria-e-gerenciamento-062012-a-112012}

Gestão do desenvolvimento do projeto do Complexo Portuário do Espadarte,
um projeto de capital da Vale S.A. (GAPON), de acordo com a metodologia
\emph{Front-End Loading} (FEL), com a elaboração, a integração e o
controle de documentos de planejamento e das ferramentas e das técnicas
da metodologia. Responsável pela integração das informações das demais
contratadas responsáveis pela execução do projeto, elaborando relatórios
e reuniões de acompanhamento.

\subsection{Engenheiro de Planejamento na Ductor Implantação de Projetos
(TÜV Rheinland) -- 08/2011 a
06/2012}\label{engenheiro-de-planejamento-na-ductor-implantauxe7uxe3o-de-projetos-tuxfcv-rheinland-082011-a-062012}

Elaboração do plano de execução do projeto (PEP), participando do
desenvolvimento e da avaliação dos projetos de obras de arte especiais
da Estrada de Ferro Vitória-Minas (EFVM), implantadas pela Vale S.A.
(GAIPG), conforme a metodologia \emph{Front-End Loading} (FEL).
Gerenciamento e fiscalização das obras de manutenção e de melhoria da
Estrada de Ferro Vitória-Minas (EFVM), implantadas pela Vale S.A.
(GAIPG), gestão do OPEX, realizando o acompanhamento dos serviços e dos
avanços físico-financeiros, além da garantia do cumprimento das
especificações, das premissas, das exigências de segurança e saúde e da
qualidade do produto. Responsável por promover as reuniões de
planejamento e as reuniões de análises de riscos das obras (análise
SWOT) com as contratadas. Análise e a validação dos cronogramas,
estruturas analíticas de projeto (EAP), curvas de progresso
físico-financeiro, histogramas de mão-de-obra e equipamentos
apresentados.

\subsection{Engenheiro Eletricista na Tozato Measurements -- 05/2007 a
07/2011}\label{engenheiro-eletricista-na-tozato-measurements-052007-a-072011}

Desenvolvimento de novos equipamentos de instrumentação e controle e o
aperfeiçoamento dos existentes, criando e executando os projetos do
\emph{hardware} (microeletrônica / microprocessamento) e do
\emph{software} (programação alto nível / aplicativos para computador),
relacionados a processos siderúrgicos. Responsável pelo aprimoramento do
sistema monitor da oscilação (produto / serviço / \emph{software}) usado
para monitoração permanente ou eventual do desempenho dos osciladores
eletromecânicos ou hidráulicos das máquinas de lingotamento contínuo em
tempo real. Responsável pelo desenvolvimento (desde projeto conceitual
até produto final) de um sistema portátil medidor de perfil, conicidade,
desgaste e deformação das faces e diagonais do molde das máquinas de
lingotamento contínuo, incluindo o equipamento com os sensores de
contato e o software que processa as informações coletadas. Participação
em projetos de implantação dos equipamentos nas plantas industriais
(siderurgias), em empresas nacionais e internacionais (Itália, Alemanha,
Romênia etc.), promovendo os ajustes personalizados para cada cliente e
acompanhando os respectivos comissionamentos.

\subsection{Estagiário na Universidade Federal de Viçosa -- 10/2005 a
04/2006}\label{estagiuxe1rio-na-universidade-federal-de-viuxe7osa-102005-a-042006}

Realização de estudo e pesquisa em soluções baseadas em
microcontroladores (eletrônica embarcada) para o desenvolvimento de um
\emph{hardware} para medição de consumo de potência (energia), incluindo
o \emph{software} embarcado. Atividade desenvolvida no departamento de
informática da universidade.

\end{document}
